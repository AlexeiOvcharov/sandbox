\documentclass[russian, utf8]{eskdtext}
\newcommand*{\No}{\textnumero}
\ESKDdepartment{Кафедра Систем Управления и Информатики}
\ESKDcompany{Университет ИТМО}
\ESKDdocName{ПОЯСНИТЕЛЬНАЯ ЗАПИСКА}
\ESKDsignature{КСУИ.207.435.001 ПЗ}
\ESKDauthor{Дема Н.Ю.}

\usepackage[hidelinks]{hyperref}

\usepackage{tocloft}
\renewcommand{\cftaftertoctitle}{\hfill}
\renewcommand{\cfttoctitlefont}{\hspace{7cm}\Large\bfseries}	%KOSTIL'
\renewcommand{\cftaftertoctitle}{\hfill}
\renewcommand{\cftdot}{.}
\renewcommand{\cftsecleader}{\cftdotfill{\cftdotsep}} % for sections

\usepackage{wrapfig}
\usepackage{graphicx}
%\graphicspath{{pictures/}}
\DeclareGraphicsExtensions{.pdf, .jpg, .png}

\usepackage{caption}
\captionsetup[table]{singlelinecheck=false}		% Заголовок таблиц слева
\captionsetup[table]{aboveskip=2pt,belowskip=2pt}			%belowskip=6pt 	% Позиционирование заголовка

% Рисование 
\usepackage{tikz}

\begin{document}
\tableofcontents
\thispagestyle{empty}
\newpage

% Introduction ------------------------------------------------------------
{\section*{Введение}
\addcontentsline{toc}{section}{Введение}}

В данной курсовой работе, в соответствии с полученным заданием, требуется разработать модуль управления для бесколлекторного электропривода постоянного тока на базе микроконтроллера, отвечающим следующим требованиям: 
\begin{itemize}
	\item тип двигателя - 3х фазный бесколлекторный
	\item способ управления - коммутация обмоток
	\item обратная связь - датчики Холла
	\item режимы - Пуск, Останов., Выбор скорости
	\item микроконтроллер - ATMEL
	\item связь с компьютером - RS 232
	\item гальваническая развязка линии связи с компьютером
	\item питание - 12 вольт постоянного тока
\end{itemize} \par
У бесколлекторного двигателя постоянного тока (БДПТ), по сравнению с обычными двигателями постоянного тока достаточно много плюсов. Главным его достоинством является отсутствие щеточно-коллекторного узла, что сильно увеличивает надежность и время эксплуатации двигателя. Также из-за отсутствия щеточно-коллекторно узла увеличивается диапозон изменения скоростей, уменьшаются массо-габаритные показатели и увеличивается КПД, поскольку отсутствуют потери на щеточно-коллекторном узле. \par

Едиственным большим минусом, из-за которого еще не отказались от обычных двигателей постоянного тока, это достаточно большой по габаритам и сложный блок управления, регулятор. Без него нельзя запустить двигатель, поскольку для минимально работы требуется перключать фазы БДПТ в определенный момент времени.\par

Регулятор состоит из силового каскада, который коммутирует обмотки в определенный момент времени. Сам каскад управляется коммутирующим устройством, генерирующим последовательность импульсов (ЧИМ) определенной частоты. Для того, чтобы просто запустить двигатель этого достаточно. Но чаще всего необходимо регулировать скорость двигателя. Cамым простым, неточным и дешевым способом является использование датчиков Холла. Они позволяют качественней коммутировать обмотки, а также вычислять скорость двигателя. \par

Благодаря высокой надёжности и хорошей управляемости, бесколлекторные двигатели применяются в широком спектре приложений: от компьютерных вентиляторов и CD/DVD-приводов до роботов и космических ракет. Также этот тип двигателей часто используется в квадрокоптерах. Широкое применение БДПТ нашли в промышленности, особенно в системах регулирования скорости с большим диапазоном и высоким темпом пусков, остановок и реверса; авиационной технике, автомобильном машиностроении, биомедицинской аппаратуре, бытовой технике и проч. \par

\newpage
%% Functional scheme of BLDC motor ------------------------------------------
\section{Функциональная схема}
Функциональная работа системы изображена на рисунке 1.
\begin{figure} [h!]
	\centering
	\begin{tikzpicture}
		% \draw[gray, dotted] (0, 0) grid (15, 6);
		\draw[thick] (0, 2.5) -- (0, 3.5) -- (1, 3.5) -- (1, 2.5) -- (1, 2.5) -- (0, 2.5); \draw (0.5, 3) node {ЗУ};
		\draw[thick, ->] (1, 3) -- (2.75, 3) node[anchor = south east] {$\omega_d$};
		\draw[thick] (3, 3) circle (0.25); 
		\draw[thick] (2.82, 2.82) -- (3.18, 3.18); \draw[thick] (3.18, 2.82) -- (2.82, 3.18);
		\draw[fill] (3, 3) -- (3.18, 2.82) arc(315:225:0.25) -- (3, 3);
		\draw[thick, ->] (3.25, 3) -- (4.5, 3) node[anchor = south east] {$e$}; 
		\draw[thick] (4.5, 2.5) -- (4.5, 3.5) -- (6, 3.5) -- (6, 2.5) -- (4.5, 2.5); \draw (5.25, 3) node {РГ};
		\draw[thick, ->] (6, 3) -- (7, 3);
		\draw[thick] (7, 2.5) -- (7, 3.5) -- (8.5, 3.5) -- (8.5, 2.5) -- (7, 2.5); \draw (7.75, 3) node {ГН};
		\draw[thick, ->] (8.5, 3) -- (9.5, 3);
		\draw[thick] (9.5, 2.5) -- (9.5, 3.5) -- (11, 3.5) -- (11, 2.5) -- (9.5, 2.5); \draw (10.25, 3) node {СК};
		\draw[thick, ->] (11, 3) -- (12, 3);
		\draw[thick] (12, 2.5) -- (12, 3.5) -- (14, 3.5) -- (14, 2.5) -- (12, 2.5); \draw (13, 3) node {БДПТ};
		\draw[thick, ->] (14, 3) -- (15, 3) node[anchor = south east] {$\omega_a$};
		%Feedback
		\draw[thick] (13, 2.5) -- (13, 2) node[anchor = west] {\scriptsize (Ha, Hb, Hc)};
		\draw[fill] (13, 2) circle (0.07);
		\draw[thick, ->] (13, 2) -- (5.25, 2) -- (5.25, 2.5);
		\draw[thick, ->] (13, 2) -- (13, 1) -- (9, 1);
		\draw[thick] (7.5, 0.5) -- (7.5, 1.5) -- (9, 1.5) -- (9, 0.5) -- (7.5, 0.5); \draw (8.25, 1) node {КС};
		\draw[thick, ->] (7.5, 1) -- (3, 1) -- (3, 2.75) node[anchor = north west] {$\omega_a$};
	\end{tikzpicture}
	\caption{Функциональная схема}
\end{figure}

\end{document}