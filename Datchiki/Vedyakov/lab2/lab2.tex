\documentclass[a4paper, 11pt]{article}

% Set encoding and language
\usepackage[T2A]{fontenc}
\usepackage[utf8]{inputenc}
\usepackage[english, russian]{babel}

% Set paper geometry
\usepackage[top=2cm, bottom=2cm, left=2cm, right=2cm]{geometry}

% Set package for include images e.t.c
\usepackage{graphicx}

% Set package based on tikz for draw graph
\usepackage{pgfplots}
\pgfplotsset{compat=1.5}

% Set package for draw table from .dat file 
\usepackage{pgfplotstable}
\usepackage{threeparttable}
% recommended:
\usepackage{booktabs}
\usepackage{array}
\usepackage{colortbl}
\usepackage{longtable}

\pgfplotstableset{
    every head row/.style = {after row = \midrule, before row = \toprule},
    every last row/.style = {after row = \bottomrule},
}

% Include \text command
\usepackage{amsmath}

\usepackage{indentfirst}

\usepackage{float}
\usepackage{array}

%Change label separator
\usepackage{caption}
\captionsetup[table]{labelformat=simple, labelsep = endash, justification = raggedright, singlelinecheck = off}
\captionsetup[figure]{labelformat=simple, labelsep = endash, name = Рисунок}

\usepackage{subcaption}

\begin{document}

\input{/home/senserlex/homeworks/titlepage/title.tex}

\section*{Инкрементальный оптический энкодер}

\noindent
\begin{minipage}[t] {0.55\textwidth}
    \begin{figure}[H]
        \begin{tikzpicture}
            \begin{axis}[
                xlabel = {$\alpha$, град.},
                ylabel = {$N$, число импульсов},
                grid style = {dashed},
                grid = major,
                width = 0.9\textwidth,
                legend pos = north west,
            ]
                \addplot[very thick, blue, mark = *] table[x = deg, y = n1] {data/encoder.dat};
                \addplot table[x = deg, y = n2] {data/encoder.dat};
                \legend{По часовой, Против часовой}
            \end{axis}
        \end{tikzpicture}
        \caption{Передаточная характеристика}
    \end{figure}
\end{minipage}
\begin{minipage}[t] {0.4\textwidth}
    \begin{table}[H]
        \centering
        \begin{threeparttable}
            \pgfplotstabletypeset[
                columns/deg/.style = {column name = {$\alpha$}},
                columns/n1/.style = {column name = {N, по часовой}},
                columns/n2/.style = {column name = {N, против часовой}},
            ]{data/encoder.dat}
        \caption{Исходные данные}
        \end{threeparttable}
    \end{table}
\end{minipage}

\vspace{0.5cm}
Как видно из рисунка 1, харастеристика при вращении по часовой и против не изменяется. Разрешающая способность датчика можно определить: 
\begin{equation}
    N_0 = 360\frac{\sum_{i = 1}^n{\frac{N_i}{\alpha_i}}}{n} = 1001.2 \text{ импульсов/оборот}
\end{equation}

\section*{Вращающийся трансформатор}
\subsection*{Характеристика синусной обмотки при нагрузке}
Ниже представлены графики, полученные в результате не сложных вычислений, а также с синусной обмотки ВТ. Графики 2 и 3, были построены на основе таблиц 2 и 3.

\begin{figure} [h!]
    \begin{subfigure} {0.5\textwidth}
        \begin{tikzpicture}    
            \begin{axis}[
                xmin = 0, xmax = 360,
                width = 0.9\textwidth,
                xlabel = {$\alpha$, град},
                ylabel = {Вольты},
                legend pos = {south west},
            ]
                \addplot table[x = deg, y = Ea] {data/Sin303.dat};
                \addplot table[x = deg, y = Ea0] {data/Sin303.dat};
                \addplot table[x = deg, y = deltaA] {data/Sin303.dat};

                \legend{$E_A$, $E_{A0}$, $\Delta E_A$}
            \end{axis}
        \end{tikzpicture}
    \end{subfigure}
    \begin{subfigure} {0.5\textwidth}
        \begin{tikzpicture}    
            \begin{axis}[
                xmin = 0, xmax = 360,
                width = 0.9\textwidth,
                xlabel = {$\alpha$, град},
                ylabel = {$\Delta E_A\%$, \%},
                legend pos = {south west},
            ]
                \addplot table[x = deg, y = deltaAproc] {data/Sin303.dat};
            \end{axis}
        \end{tikzpicture}
    \end{subfigure}
    \caption{Характеристики синусной обмотки при нагрузке $R = 300$ Ом}
\end{figure}

\begin{figure} [h!]
    \begin{subfigure} {0.5\textwidth}
        \begin{tikzpicture}    
            \begin{axis}[
                xmin = 0, xmax = 360,
                width = 0.9\textwidth,
                xlabel = {$\alpha$, град},
                ylabel = {Вольты},
                legend pos = {south west},
            ]
                \addplot table[x = deg, y = Ea] {data/Sin500.dat};
                \addplot table[x = deg, y = Ea0] {data/Sin500.dat};
                \addplot table[x = deg, y = deltaA] {data/Sin500.dat};

                \legend{$E_A$, $E_{A0}$, $\Delta E_A$}
            \end{axis}
        \end{tikzpicture}
    \end{subfigure}
    \begin{subfigure} {0.5\textwidth}
        \begin{tikzpicture}    
            \begin{axis}[
                xmin = 0, xmax = 360,
                width = 0.9\textwidth,
                xlabel = {$\alpha$, град},
                ylabel = {$\Delta E_A\%$, \%},
                legend pos = {south west},
            ]
                \addplot table[x = deg, y = deltaAproc] {data/Sin500.dat};
            \end{axis}
        \end{tikzpicture}
    \end{subfigure}
    \caption{Характеристики синусной обмотки при нагрузке $R = 500$ Ом}
\end{figure}

\begin{minipage} {0.5\textwidth}
    \begin{table}[H]
        \centering
        \begin{threeparttable}
            \pgfplotstabletypeset[
                columns/deg/.style = {column name = {$\alpha$}},
                columns/Ea/.style = {column name = {$E_A$}},
                columns/Ea0/.style = {column name = {$E_{A0}$}},
                columns/deltaA/.style = {column name = {$\Delta E_{A}$}},
                columns/deltaAproc/.style = {column name =  {$\Delta E_{A}\%$}},
            ]{data/Sin303.dat}
        \caption{Данные при $R = 300$ Ом}
        \end{threeparttable}
    \end{table}
\end{minipage}
\begin{minipage} {0.5\textwidth}
    \begin{table}[H]
        \centering
        \begin{threeparttable}
            \pgfplotstabletypeset[
                columns/deg/.style = {column name = {$\alpha$}},
                columns/Ea/.style = {column name = {$E_A$}},
                columns/Ea0/.style = {column name = {$E_{A0}$}},
                columns/deltaA/.style = {column name = {$\Delta E_{A}$}},
                columns/deltaAproc/.style = {column name =  {$\Delta E_{A}\%$}},
            ]{data/Sin500.dat}
        \caption{Данные при $R = 500$ Ом}
        \end{threeparttable}
    \end{table}
\end{minipage}

\subsection*{Характеристика косинусной обмотки при нагрузке}
Ниже представлены графики, полученные в результате не сложных вычислений, а также с синусной обмотки ВТ. Графики 4 и 5, были построены на основе таблиц 4 и 5.

\begin{figure} [h!]
    \begin{subfigure} {0.5\textwidth}
        \begin{tikzpicture}    
            \begin{axis}[
                xmin = 0, xmax = 360,
                width = 0.9\textwidth,
                xlabel = {$\alpha$, град},
                ylabel = {Вольты},
                legend pos = {south west},
            ]
                \addplot table[x = deg, y = Eb] {data/Cos315.dat};
                \addplot table[x = deg, y = Eb0] {data/Cos315.dat};
                \addplot table[x = deg, y = deltaB] {data/Cos315.dat};

                \legend{$E_B$, $E_{B0}$, $\Delta E_B$}
            \end{axis}
        \end{tikzpicture}
    \end{subfigure}
    \begin{subfigure} {0.5\textwidth}
        \begin{tikzpicture}    
            \begin{axis}[
                xmin = 0, xmax = 360,
                width = 0.9\textwidth,
                xlabel = {$\alpha$, град},
                ylabel = {$\Delta E_B\%$, \%},
                legend pos = {south west},
            ]
                \addplot table[x = deg, y = deltaBproc] {data/Cos315.dat};
            \end{axis}
        \end{tikzpicture}
    \end{subfigure}
    \caption{Характеристики синусной обмотки при нагрузке $R = 300$ Ом}
\end{figure}

\begin{figure} [h!]
    \begin{subfigure} {0.5\textwidth}
        \begin{tikzpicture}    
            \begin{axis}[
                xmin = 0, xmax = 360,
                width = 0.9\textwidth,
                xlabel = {$\alpha$, град},
                ylabel = {Вольты},
                legend pos = {south west},
            ]
                \addplot table[x = deg, y = Eb] {data/Cos510.dat};
                \addplot table[x = deg, y = Eb0] {data/Cos510.dat};
                \addplot table[x = deg, y = deltaB] {data/Cos510.dat};

                \legend{$E_B$, $E_{B0}$, $\Delta E_B$}
            \end{axis}
        \end{tikzpicture}
    \end{subfigure}
    \begin{subfigure} {0.5\textwidth}
        \begin{tikzpicture}    
            \begin{axis}[
                xmin = 0, xmax = 360,
                width = 0.9\textwidth,
                xlabel = {$\alpha$, град},
                ylabel = {$\Delta E_B\%$, \%},
                legend pos = {south west},
            ]
                \addplot table[x = deg, y = deltaBproc] {data/Cos510.dat};
            \end{axis}
        \end{tikzpicture}
    \end{subfigure}
    \caption{Характеристики косинусной обмотки при нагрузке $R = 500$ Ом}
\end{figure}

\begin{minipage} {0.5\textwidth}
    \begin{table}[H]
        \centering
        \begin{threeparttable}
            \pgfplotstabletypeset[
                columns/deg/.style = {column name = {$\alpha$}},
                columns/Eb/.style = {column name = {$E_B$}},
                columns/Eb0/.style = {column name = {$E_{B0}$}},
                columns/deltaB/.style = {column name = {$\Delta E_{B}$}},
                columns/deltaBproc/.style = {column name =  {$\Delta E_{B}\%$}},
            ]{data/Cos315.dat}
        \caption{Данные при $R = 300$ Ом}
        \end{threeparttable}
    \end{table}
\end{minipage}
\begin{minipage} {0.5\textwidth}
    \begin{table}[H]
        \centering
        \begin{threeparttable}
            \pgfplotstabletypeset[
                columns/deg/.style = {column name = {$\alpha$}},
                columns/Eb/.style = {column name = {$E_B$}},
                columns/Eb0/.style = {column name = {$E_{B0}$}},
                columns/deltaB/.style = {column name = {$\Delta E_{B}$}},
                columns/deltaBproc/.style = {column name =  {$\Delta E_{B}\%$}},
            ]{data/Cos510.dat}
        \caption{Данные при $R = 500$ Ом}
        \end{threeparttable}
    \end{table}
\end{minipage}

\subsection*{Характеристики ВТ при вторичном симметрировании}
\begin{figure} [h!]
    \begin{subfigure} {0.5\textwidth}
        \begin{tikzpicture}    
            \begin{axis}[
                xmin = 0, xmax = 315,
                width = 0.9\textwidth,
                xlabel = {$\alpha$, град},
                ylabel = {Вольты},
                legend pos = {south west},
            ]
                \addplot table[x = deg, y = Ea] {data/SecondSimetr.dat};
                \addplot table[x = deg, y = Ea0] {data/SecondSimetr.dat};
                \addplot table[x = deg, y = deltaA] {data/SecondSimetr.dat};

                \legend{$E_A$, $E_{A0}$, $\Delta E_A$}
            \end{axis}
        \end{tikzpicture}
    \end{subfigure}
    \begin{subfigure} {0.5\textwidth}
        \begin{tikzpicture}    
            \begin{axis}[
                xmin = 0, xmax = 315,
                width = 0.9\textwidth,
                xlabel = {$\alpha$, град},
                ylabel = {$\Delta E_A\%$, \%},
                legend pos = {south west},
            ]
                \addplot table[x = deg, y = deltaAproc] {data/SecondSimetr.dat};
            \end{axis}
        \end{tikzpicture}
    \end{subfigure}
    \caption{Характеристики синусной обмотки при нагрузке $R = 1$ кОм}
\end{figure}
\newpage
\begin{table}[h!]
    \centering
    \begin{threeparttable}
        \pgfplotstabletypeset[
            columns/deg/.style = {column name = {$\alpha$}},
            columns/Eb/.style = {column name = {$E_B$}},
            columns/Eb0/.style = {column name = {$E_{B0}$}},
            columns/deltaB/.style = {column name = {$\Delta E_{B}$}},
            columns/deltaBproc/.style = {column name =  {$\Delta E_{B}\%$}},
        ]{data/Cos510.dat}
        \caption{Данные при $R = 1$ кОм}
    \end{threeparttable}
\end{table}

\section*{Потенциометрический датчик угла поворота}
\subsection*{Характеристика холостого хода}

\begin{minipage}[t] {0.5\textwidth}
    \begin{figure}[H]
        \begin{tikzpicture}
            \begin{axis}[
                width = 0.9\textwidth,
                xmin = 0, xmax = 360,
                xlabel = {$\alpha$, град},
                ylabel = {$U_\text{вых}$, В},
                legend pos = {south west},
            ]
                \addplot table[x = deg1, y = U1] {data/PotensTrans.dat};
                \addplot table[x = deg2, y = U2] {data/PotensTrans.dat};

                \legend{По часовой, Против часовой}
            \end{axis}
        \end{tikzpicture}
        \caption{Характеристика хаолостого хода}
    \end{figure}
\end{minipage}
\begin{minipage}[t] {0.4\textwidth}
    \begin{table}[H]
        \centering
        \begin{threeparttable}
            \pgfplotstabletypeset[
                columns/deg1/.style = {column name = {$\alpha$}},
                columns/U1/.style = {column name = {$U_\text{вых}$}},
                columns/deg2/.style = {column name = {$\alpha$}},
                columns/U2/.style = {column name = {$U_\text{вых}$}},
            ]{data/PotensTrans.dat}
        \caption{Исходные данные}
        \end{threeparttable}
    \end{table}
\end{minipage}

\newpage
\subsection*{Характеристики однотактного потенциометрического
датчика угла поворота под нагрузкой}
\begin{figure}[h!]
    \centering
    \begin{tikzpicture}
        \begin{axis}[
            width = 0.7\textwidth,
            xmin = 0, xmax = 360, ymin = 0,
            xlabel = {$\alpha$, град},
            ylabel = {$U_\text{вых}$, В},
            legend pos = {north east},
        ]
            \addplot[blue, mark = *, dashed] table[x = deg1, y = U1n] {data/PotensTransR.dat};
            \addplot[blue, mark = *, dashed] table[x = deg2, y = U1p] {data/PotensTransR.dat};
            \addplot[black, mark = o, dotted, thick] table[x = deg1, y = U2n] {data/PotensTransR.dat};
            \addplot[black, mark = o, dotted, thick] table[x = deg2, y = U2p] {data/PotensTransR.dat};
            \addplot[red, mark = x] table[x = deg1, y = U3n] {data/PotensTransR.dat};
            \addplot[red, mark = x] table[x = deg2, y = U3p] {data/PotensTransR.dat};

            \legend{R = 250, , R = 500, ,R = 1000}
        \end{axis}
    \end{tikzpicture}
    \caption{Характеристики при различных сопротивлениях}
\end{figure}
\begin{table}[h!]
    \centering
    \begin{threeparttable}
        \pgfplotstabletypeset[
            columns/deg1/.style = {column name = {$\alpha$}},
            columns/U1n/.style = {column name = {$U_\text{вых}$}},
            columns/deg2/.style = {column name = {$\alpha$}},
            columns/U1p/.style = {column name = {$U_\text{вых}$}},
            columns/deg3/.style = {column name = {$\alpha$}},
            columns/U2n/.style = {column name = {$U_\text{вых}$}},
            columns/deg4/.style = {column name = {$\alpha$}},
            columns/U2p/.style = {column name = {$U_\text{вых}$}},
            columns/deg5/.style = {column name = {$\alpha$}},
            columns/U3n/.style = {column name = {$U_\text{вых}$}},
            columns/deg6/.style = {column name = {$\alpha$}},
            columns/U3p/.style = {column name = {$U_\text{вых}$}},
        ]{data/table.dat}
    \caption{Исходные данные}
    \end{threeparttable}
\end{table}

\section*{Вывод}
В данной работе мы исследовали характеристики трех датчиков углового перемещения: потенциометрического, вращающегося трансформатора, инкриментального энкодера. У потенциометрического датчика характеристика при повороте по часовой и против часовой отличается. При увеличении сопротивления возростает нелинейность харатеристики. Во вращающемся трансформаторе при увеличении сопротивления - увеличиваетсся выходоне нарпяжение. Синусная и косинусная обмотки ведут себя одинаково.
\end{document}
