\documentclass[a4paper, 11pt]{article}

% Set encoding and language
\usepackage[T2A]{fontenc}
\usepackage[utf8]{inputenc}
\usepackage[english, russian]{babel}

% Set paper geometry
\usepackage[top=2cm, bottom=2cm, left=2cm, right=2cm]{geometry}

% Set package for include images e.t.c
\usepackage{graphicx}

% Set package based on tikz for draw graph
\usepackage{pgfplots}
\pgfplotsset{compat=1.5}

% Set package for draw table from .dat file 
\usepackage{pgfplotstable}
% recommended:
\usepackage{booktabs}
\usepackage{array}
\usepackage{colortbl}
\usepackage{longtable}

% set thousands separator from "," to space 
\pgfkeys{/pgf/number format/.cd,
    set thousands separator={\text{ }}
}

% Include \text command
\usepackage{amsmath}

\usepackage{float}
\usepackage{array}

\newcolumntype{R}[1]{>{\raggedleft\arraybackslash}p{#1}}
\newcolumntype{L}[1]{>{\raggedright\arraybackslash}p{#1}}
\newcolumntype{C}[1]{>{\raggedcenter\arraybackslash}p{#1}}

\begin{document}

\input{/home/senserlex/homeworks/titlepage/title.tex}

\section*{Передаточная характеристика холостого хода тахогенератора постоянного тока.}
На рисунке 1 представлен график зависимости $U_\text{вых}(n)$. Как видно соблюдается линейная зависимость между угловой скоростью и выходным напряжением тахогенератора.

\begin{minipage}[c]{0.25\textwidth}
\begin{table}[H]
\pgfplotstabletypeset[
    columns/n/.style = {
        column name = {$n\text{, об/мин}$},
    },
    columns/u/.style = {
        column name = {$u_{\text{вых}}\text{, в}$},
    },
    empty cells with={--}, % replace empty cells with ’--’
    every head row/.style={before row=\toprule,after row=\midrule},
    every last row/.style={after row=\bottomrule}
]{../src/data1.dat}
\caption{Данные при R = 0 Ом}
\end{table}
\end{minipage}
\begin{minipage}[c]{0.7\textwidth}
\begin{figure}[H]
    \centering
    \begin{tikzpicture}
        \begin{axis}[
            width = 11cm, height = 7cm,
            xlabel = {$n\text{, об/мин}$},
            ylabel = {$U_{\text{вых}}\text{, В}$},
            ymin = 0, ymax = 20,
            minor y tick num = 1,
            ymajorgrids=true,
            xmajorgrids=true,
            grid style=dashed,
        ]
            \addplot table {../src/data1.dat};
        \end{axis}
    \end{tikzpicture}
    \caption{Передаточная характеристика.}
\end{figure}
\end{minipage}


\section*{Передаточная характеристика холостого хода тахогенератора постоянного тока при нагрузке.}

Далее в таблице 2 изложены данные по опытам при изменении сопротивления нагрузки $R_\text{н}$ тахогенератора. А на рисунке 2 нарисованы графики, соответствующие данным в этой таблице.

\begin{figure}[h!]
    \centering
    \begin{tikzpicture}
        \begin{axis}[
            width = 17cm, height = 10cm,
            xlabel = {$n\text{, об/мин}$},
            ylabel = {$U_{\text{вых}}$, В},
            ymin = 0, ymax = 20,
            minor y tick num = 1,
            ymajorgrids = true,
            xmajorgrids = true,
            legend pos = north west,
        ]
            \addplot table {../src/data1.dat};
            \addplot table {../src/data2.dat};
            \addplot table {../src/data3.dat};
            \addplot table {../src/data4.dat};
            \addplot table {../src/data5.dat};
            \addplot table {../src/data6.dat};
            \addplot table {../src/data7.dat};
            \addplot table {../src/data8.dat};

            \legend{$R = 0$ Ом,$R = 99.5$ Ом,$R = 496$ Ом,$R = 998$ Ом,$R = 2930$ Ом,$R = 5000$ Ом,$R = 7000$ Ом, $R = 10000$ Ом}
        \end{axis}
    \end{tikzpicture}
    \caption{Передаточная характеристика при различной нагрузке.}
\end{figure}

\newpage

\begin{table}[h!]
\hspace{-0.75cm}
\pgfplotstabletypeset[
    every head row/.style={
        before row={%
        \hline
        \multicolumn{2}{| c |}{R = 99.5} & \multicolumn{2}{ c |}{R = 469} & \multicolumn{2}{ c |}{R = 998} & \multicolumn{2}{ c }{R = 2930} & \multicolumn{2}{| c |}{R = 5000} & \multicolumn{2}{ c |}{R = 7000} & \multicolumn{2}{ c |}{R = 10000} \\
        \hline
        },
        after row=\hline,
    },
    every last row/.style={
        after row=\hline},
    columns/n2/.style = {
        column name = {$n$},
        string replace={0}{},
        column type = {| c },
    },
    columns/U2/.style = {
        column name = {$U$},
        string replace={0}{},
        column type = {| c },
    },
    columns/n3/.style = {
        column name = {$n$},
        string replace={0}{},
        column type = {| c },
    },
    columns/U3/.style = {
        column name = {$U$},
        string replace={0}{},
        column type = {| c },
    },
    columns/n4/.style = {
        column name = {$n$},
        string replace={0}{},
        column type = {| c },
    },
    columns/U4/.style = {
        column name = {$U$},
        string replace={0}{},
        column type = {| c },
    },
    columns/n5/.style = {
        column name = {$n$},
        string replace={0}{},
        column type = {| c },
    },
    columns/U5/.style = {
        column name = {$U$},
        string replace={0}{},
        column type = {| c },
    },
    columns/n6/.style = {
        column name = {$n$},
        string replace={0}{},
        column type = {| c },
    },
    columns/U6/.style = {
        column name = {$U$},
        string replace={0}{},
        column type = {| c },
    },
    columns/n7/.style = {
        column name = {$n$},
        string replace={0}{},
        column type = {| c },
    },
    columns/U7/.style = {
        column name = {$U$},
        string replace={0}{},
        column type = {| c },
    },
    columns/n8/.style = {
        column name = {$n$},
        string replace={0}{},
        column type = {| c },
    },
    columns/U8/.style = {
        column name = {$U$},
        string replace={0}{},
        column type = {| c |},
    },
    empty cells with={--}, % replace empty cells with ’--’
]{../src/res.dat}
\caption{Исходные данные при различной нагрузке.}
\end{table}

\section*{Определение частоты вращения при помощи инкрементального энкодера.}

\begin{table} [h!]
	\centering
	\begin{tabular} {|p{3cm}|p{4cm}|p{3cm}|p{5cm}|}
		\hline
		Частота вращения по тахометру & Частота появления импульсов энкодера & Разрешающая способность & Частота вращения ротора двигателя вычисленная по энкодеру \\
		об/мин & Гц & отсчетов/оборот & об/мин \\ \hline
		500 & 15.25 & 1.83 & 500 \\
		420 & 14.3 & 2.0428571 & 420 \\
		350 & 12.25 & 2.1 & 350 \\
		300 & 10.5 & 2.1 & 300 \\
		240 & 8 & 2 & 240 \\ \hline
	\end{tabular}
	\caption{Данные с фазы A энкодера.}
\end{table}

\begin{table} [h!]
	\begin{tabular} {|p{3cm}|p{4cm}|p{3cm}|p{5cm}|}
		\hline
		Частота вращения по тахометру & Частота появления импульсов энкодера & Разрешающая способность & Частота вращения ротора двигателя вычисленная по энкодеру \\
		об/мин & Гц & отсчетов/оборот & об/мин \\ \hline
		545 & 18.7 & 2.059 & 545 \\
		500 & 17.5 & 2.1 & 500 \\ 
		370 & 12.6 & 2.043 & 370 \\
		330 & 11.3 & 2.055 & 330 \\
		230 & 8 & 2.087 & 230 \\ \hline
	\end{tabular}
	\caption{Данные с фазы B энкодера.}
\end{table}

Разрешающая способность была найдена по следующей формуле (1). Здесь $\n$ - частота вращения по тахометру, $\nu$ - частота появления импульсов.
\begin{equation}
	r = \frac{\nu60}{n}
\end{equation}
По полученным данным можно сделать вывод, что разрешающая способность данного энкодера равна 2.
\newpage
\section*{Сравнение полученных характеристик с паспортными данными.}
Сравним полученную характеристику при R = 10000 Ом с характеристикой по паспортным данным. Как видно из рисунка 3 крутизна характеристики эспериментальных данных $k_\text{э} = 6.6 \frac{\text{мВ}}{\text{об/мин}}$ и паспортных $k = 4 \frac{\text{мВ}}{\text{об/мин}}$ отличаются.
\begin{figure}[h!]
    \centering
    \begin{tikzpicture}
        \begin{axis}[
            width = 17cm, height = 7cm,
            ymin = 0, ymax = 20,
            xmin = 0, xmax = 3000,
            ymajorgrids = true,
            grid style = dashed,
            legend pos = north west,
        ]
            \addplot table {../src/data8.dat};
            \addplot [red, domain=200:2800]{0.004*x};
            \legend{Экспериментальная, Паспортные данные}
        \end{axis}
    \end{tikzpicture}
    \caption{Сравнительные данные.}
\end{figure}


\section*{Выводы.}
В данной работе мы ознакомились с тахогенератором ТГП-3, получили его характеристику при нулевой и ненулевой нагрузке. Как видно из графиков 1, 2 и 3 - нелинейностей характеристик не наблюдается, поскольку угловая скорость тахогенератора не достасточно высока. \par
Также мы рассчитали разрешающую способность инкрементального энкодера E50S8.
\end{document}
